\documentclass[10pt,a4paper]{article}
\usepackage{tikz}
\usetikzlibrary{positioning}
\usepackage[spanish]{babel}

\begin{document}
	
	\begin{figure}[h]
		\centering
		\begin{tikzpicture}[
			folder/.style={
				rectangle,
				minimum height=0.6cm,
				inner sep=4pt,
				text width=3.2cm,
				draw=blue!60,
				fill=blue!10,
				rounded corners=2pt,
				font=\small\sffamily,
				align=left
			},
			root/.style={
				rectangle,
				minimum height=0.7cm,
				inner sep=4pt,
				text width=3.2cm,
				draw=red!60,
				fill=red!5,
				rounded corners=2pt,
				font=\small\sffamily\bfseries,
				align=left
			}
			]
			% Raíz C:
			\node[root] (croot) {C:.};
			
			% Primera columna - Carpetas principales
			\node[folder, below=0.2cm of croot] (contacts) {Contacts};
			\node[folder, below=0.1cm of contacts] (desktop) {Desktop};
			\node[folder, below=0.1cm of desktop] (sistemas) {Sistemas};
			\node[folder, below=0.1cm of sistemas] (documents) {Documents};
			\node[folder, below=0.1cm of documents] (downloads) {Downloads};
			\node[folder, below=0.1cm of downloads] (favorites) {Favorites};
			\node[folder, below=0.1cm of favorites] (links) {Links};
			\node[folder, below=0.1cm of links] (sitiosespana) {Sitios for España};
			\node[folder, below=0.1cm of sitiosespana] (sitiosmicrosoft) {Sitios web de Microsoft};
			\node[folder, below=0.1cm of sitiosmicrosoft] (sitiosmsn) {Sitios web de MSN};
			\node[folder, below=0.1cm of sitiosmsn] (windowslive) {Windows Live};
			
			% Segunda columna - Carpetas adicionales (al mismo nivel)
			\node[folder, right=0.5cm of contacts] (links2) {Links};
			\node[folder, below=0.1cm of links2] (music) {Music};
			\node[folder, below=0.1cm of music] (pictures) {Pictures};
			\node[folder, below=0.1cm of pictures] (savedgames) {Saved Games};
			\node[folder, below=0.1cm of savedgames] (searches) {Searches};
			\node[folder, below=0.1cm of searches] (videos) {Videos};
			
			% Líneas conectivas desde la raíz
			\draw (croot.south) -- (contacts.north);
			\draw (contacts.south) -- (desktop.north);
			\draw (desktop.south) -- (sistemas.north);
			\draw (sistemas.south) -- (documents.north);
			\draw (documents.south) -- (downloads.north);
			\draw (downloads.south) -- (favorites.north);
			\draw (favorites.south) -- (links.north);
			\draw (links.south) -- (sitiosespana.north);
			\draw (sitiosespana.south) -- (sitiosmicrosoft.north);
			\draw (sitiosmicrosoft.south) -- (sitiosmsn.north);
			\draw (sitiosmsn.south) -- (windowslive.north);
			
			% Líneas para la segunda columna (conectadas también a la raíz)
			\draw (croot.east) -- ++(0.5,0) |- (links2.west);
			\draw (links2.south) -- (music.north);
			\draw (music.south) -- (pictures.north);
			\draw (pictures.south) -- (savedgames.north);
			\draw (savedgames.south) -- (searches.north);
			\draw (searches.south) -- (videos.north);
			
			% Información adicional (como en el comando tree)
			\node[below=0.5cm of windowslive, font=\scriptsize\itshape] (info1) {El número de serie del volumen es C40A-EB9C};
			\node[below=0.1cm of info1, font=\scriptsize\itshape] (info2) {Estado de rutas de carpetas};
			
		\end{tikzpicture}
		\caption{Estructura de carpetas de Windows - Comando \texttt{tree}}
		\label{fig:windows_tree_structure}
	\end{figure}
	
\end{document}