\documentclass[11pt]{article}

\usepackage[T1]{fontenc}
\usepackage{minted2}
\usepackage{shellesc}
% minted: syntax highlighting via Pygments

\usepackage[utf8]{inputenc}      % Codificación UTF-8
\usepackage[T1]{fontenc}         % Acentos y caracteres especiales
\usepackage[spanish]{babel}      % Idioma español
\usepackage{lmodern}             % Fuente moderna y clara
\usepackage{xcolor}              % Colores personalizados
\usepackage{geometry}            % Márgenes
\geometry{margin=2.5cm}

\definecolor{bg}{rgb}{0.95,0.95,0.95}
\definecolor{keyword}{rgb}{0.13,0.13,1}
\definecolor{comment}{rgb}{0.25,0.5,0.35}
\definecolor{string}{rgb}{0.58,0,0.13}
\definecolor{bgdark}{HTML}{2D2D2D}

\usepackage{courier}
% Opciones globales (opcional)
\setminted{
	frame=lines,                 % Marco superior e inferior
	framesep=2mm,                % Espaciado del marco
	baselinestretch=1.1,         % Espaciado entre líneas
	fontsize=\small,     		 % Tamaño de fuente
	linenos,                     % Números de línea
	breaklines,                  % Saltar líneas largas
	bgcolor=gray!5,              % Fondo gris claro
	tabsize=4,                   % Tamaño de tabulación
	autogobble,                  % Ajusta indentación automáticamente
	xleftmargin=1em,             % Margen izquierdo,
	fontfamily=pcr
}

\begin{document}
	
	\section*{Ejemplo con minted}
	
	Aquí un fragmento en Python:
	
	\begin{minted}[bgcolor=bg, style=solarized-light]{python}
		import sys
		import cv2 # image management
		def saludo(nombre):
			print(f"Bienvenido {nombre}!")
		
		def operaciones():
			return saludo() + "ve a caminar"
			
		if __name__ == "__main__":
			saludo("Mundo")
		
	\end{minted}
	
	\section*{Other example}
	\begin{minted}[bgcolor=bgdark, style=monokai]{python}
		"""
			función para calcular la sería fibonnaci
		"""
		def fib(n):
		a, b = 0, 1
		while a < n:
			yield a
			a, b = b, a + b
		print(list(fib(10)))
	\end{minted}
	
	\section*{Example with MATLAB}
	\begin{minted}[bgcolor=bg, style=friendly]{matlab}
		x1 = 240;
		x2 = 40;
		x3 = x1 + x2;
		
		% Agregando valores al arreglo
		disp(["El valor de x3 es: ", x3]);
	\end{minted}
	
	\begin{listing}[H]
		\begin{minted}{c}
			#include <stdio.h>
			
			int main() {
				printf("Hello, World!\n");
				return 0;
			}
		\end{minted}
		\caption{Programa en C básico}
		\label{lst:c-program}
	\end{listing}
	
	\ifcase\ShellEscapeStatus
	Shell escape **no está habilitado**\or
	Shell escape **completo**\or
	Shell escape **restringido**
	\fi
\end{document}
