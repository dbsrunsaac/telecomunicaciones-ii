%----------------------------------------------------
% DOCUMENTO BASE LATEX CON MINTED
%----------------------------------------------------
\documentclass[12pt,a4paper]{article}

%----------------------------------------------------
% PAQUETES BÁSICOS
%----------------------------------------------------
\usepackage[utf8]{inputenc}      % Codificación UTF-8
\usepackage[T1]{fontenc}         % Acentos y caracteres especiales
\usepackage[spanish]{babel}      % Idioma español
\usepackage{lmodern}             % Fuente moderna y clara
\usepackage{xcolor}              % Colores personalizados
\usepackage{geometry}            % Márgenes
\geometry{margin=2.5cm}

%----------------------------------------------------
% CONFIGURACIÓN DE MINTED
%----------------------------------------------------
\usepackage{minted}              % Paquete para código con resaltado
\setminted{
	frame=lines,                 % Marco superior e inferior
	framesep=2mm,                % Espaciado del marco
	baselinestretch=1.1,         % Espaciado entre líneas
	fontsize=\small,             % Tamaño de fuente
	linenos,                     % Números de línea
	breaklines,                  % Saltar líneas largas
	bgcolor=gray!5,              % Fondo gris claro
	tabsize=4,                   % Tamaño de tabulación
	autogobble,                  % Ajusta indentación automáticamente
	xleftmargin=1em              % Margen izquierdo
}

%----------------------------------------------------
% OPCIONAL: COLORES PERSONALIZADOS PARA MINTED
%----------------------------------------------------
\definecolor{bg}{rgb}{0.95,0.95,0.95}
\definecolor{keyword}{rgb}{0.13,0.13,1}
\definecolor{comment}{rgb}{0.25,0.5,0.35}
\definecolor{string}{rgb}{0.58,0,0.13}

% Puedes usar estos colores con minted styles o crear tus propios estilos

%----------------------------------------------------
% TÍTULO DEL DOCUMENTO
%----------------------------------------------------
\title{Ejemplo de Documento con Código Fuente usando Minted}
\author{Tu Nombre}
\date{\today}

%----------------------------------------------------
% INICIO DEL DOCUMENTO
%----------------------------------------------------
\begin{document}
	
	\maketitle
	\tableofcontents
	\newpage
	
	%----------------------------------------------------
	% SECCIÓN 1: INTRODUCCIÓN
	%----------------------------------------------------
	\section{Introducción}
	
	En este documento se muestra cómo insertar código fuente con formato profesional usando el paquete \texttt{minted}.  
	Recuerda compilar con el siguiente comando:
	
	\begin{verbatim}
		pdflatex -shell-escape archivo.tex
	\end{verbatim}
	
	%----------------------------------------------------
	% SECCIÓN 2: EJEMPLOS DE CÓDIGO
	%----------------------------------------------------
	\section{Ejemplos de código}
	
	\subsection{Código en Python}
	\begin{minted}[bgcolor=bg, style=friendly]{python}
		def saludo(nombre):
		print(f"Hola, {nombre}!")
		
		if __name__ == "__main__":
		saludo("Bremdow")
	\end{minted}
	
	\subsection{Código en C}
	\begin{minted}[style=monokai]{c}
		#include <stdio.h>
		
		int main() {
			printf("Hola, mundo!\n");
			return 0;
		}
	\end{minted}
	
	\subsection{Código en MATLAB}
	\begin{minted}[style=borland]{matlab}
		x = 0:0.1:2*pi;
		y = sin(x);
		plot(x, y);
		title('Señal Senoidal');
		xlabel('Tiempo');
		ylabel('Amplitud');
	\end{minted}
	
	%----------------------------------------------------
	% SECCIÓN 3: CÓDIGO EN BLOQUES CON TÍTULO
	%----------------------------------------------------
	\section{Código dentro de un entorno con título}
	
	\begin{listing}[H]
		\caption{Ejemplo de función en C#}
		\begin{minted}[bgcolor=bg, linenos, style=colorful]{csharp}
			public static void Saludo(string nombre)
			{
				Console.WriteLine($"Hola, {nombre}!");
			}
		\end{minted}
	\end{listing}
	
	%----------------------------------------------------
	% SECCIÓN 4: CÓDIGO EN LÍNEA
	%----------------------------------------------------
	\section{Código en línea}
	
	Puedes insertar código dentro del texto como:
	\mintinline{python}{print("Hola Mundo")}
	
	O incluso en C: \mintinline{c}{printf("Hola Mundo");}
	
	%----------------------------------------------------
	% FIN DEL DOCUMENTO
	%----------------------------------------------------
\end{document}
